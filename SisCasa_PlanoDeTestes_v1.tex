\documentclass[12pt]{article}

\usepackage[brazilian]{babel}
\usepackage{indentfirst}
\usepackage{setspace}
\usepackage{graphicx} % Required for inserting images
\graphicspath{ {./Imagens/} }
\usepackage{mathtools}
\usepackage{hyperref}
\usepackage{pdfpages}
\usepackage[a4paper, left=3cm, right=2cm, top=2cm, bottom=1.95cm]{geometry}
\usepackage{float}
\usepackage{multirow}
\usepackage[version=3]{mhchem}
\usepackage{tikz}
\usepackage{tkz-fct}
\usepackage{verbatim}
\usepackage{pgfplots}
\usepackage{amsmath}
\pgfplotsset{compat=1.18, width=0.75\linewidth}
\usepackage{booktabs}
\usepackage{circuitikz}
\usepackage{siunitx}
\usepackage{gensymb}
\usepackage{caption}
\usepackage{multirow}
\usepackage{subcaption}
\usepackage{placeins}
\usepackage{tikz}
\usepackage{xcolor}
\usepackage{listings}
\usetikzlibrary{calc}
\usetikzlibrary{positioning}
\usetikzlibrary{matrix}

\hypersetup{
    colorlinks=true,
    linkcolor=black,
    filecolor=magenta,      
    urlcolor=blue,
    }
    
\urlstyle{same}

\newcommand\Tau{\mathrm{T}}
\newcommand\Epsilon{\mathcal{E}}
\newcommand\minus{-}
\newcommand{\olsi}[1]{\,\overline{\!{#1}}}

\begin{document}
\renewcommand{\arraystretch}{1.5}
\setstretch{1.3}

\thispagestyle{empty}
\begin{figure}[h]
\centering
\includegraphics[width=6cm]{images/simbolo.png}
\end{figure}
\vspace{1cm}
\begin{center}

{\vspace{5px}}
    {\bf \LARGE CSI-28 -- LAB 05}\\
\end{center}
\begin{center}
    {\bf \LARGE Testes}\\[2cm]
\end{center}
\begin{center}  
    {\large \bf Alunos}: \\[0.1cm]
    {\large Daniel da Silva Sahadi\\Diogo Bueno Rodrigues\\Luiz Felipe de Brito Ramos\\Lucas Ribeiro do Rego Barros\\Thiago Galante Pereira}\\[1cm]
    {\large \bf Turma: } \\[0.1cm]
    {COMP 27}\\[1cm]
    {\large \bf Professores: } \\[0.1cm]
    {\large Karla Donato Fook \& Johnny Cardoso Marques}\\ [8cm]
    {\bf Instituto Tecnológico de Aeronáutica - ITA }\\[1cm]
\end{center}

\newpage

\tableofcontents
\pagenumbering{Roman}
\setcounter{page}{0}

\newpage
\pagenumbering{arabic}
\setcounter{page}{1}

\newpage

\section{Introdução}

Este documento apresenta o Plano de Testes para o sistema SisCasa (Sistema Administrativo da Casa do Aconchego), definindo estratégias, técnicas, métodos e casos de teste. O escopo dos testes abrange funcionalidades implementadas, validação de integridade de dados, interface do usuário, integração entre componentes e requisitos não-funcionais básicos.

\section{Níveis de Teste de Desenvolvimento}

\subsection{Testes de Unidade}

\textbf{Objetivo}: Validar componentes individuais isoladamente.

\subsubsection{Backend}
\begin{itemize}
    \item Função \texttt{getLookupId}: busca de IDs, tratamento de nulos, valores não encontrados
    \item Endpoint POST /api/beneficiarios: validação, geração de cadastro, inserções, transações
    \item Endpoint GET /api/beneficiarios: consulta, formatação de dados relacionados
    \item Endpoints de lookup: tipos-beneficio, cidades, racas, religioes, hospitais, graus-parentesco
\end{itemize}

\subsubsection{Frontend}
\begin{itemize}
    \item \textbf{FormularioBeneficiarios}: renderização, validação, adição/remoção de responsáveis e família, submissão
    \item \textbf{ListaBeneficiarios}: renderização, busca, expansão de detalhes, filtros
    \item \textbf{CestasBasicas}: exibição de estoque e histórico
\end{itemize}

\subsection{Testes de Integração}

\textbf{Objetivo}: Validar comunicação entre componentes.

\subsubsection{Backend-Banco de Dados}
\begin{itemize}
    \item Conexão e pool de conexões
    \item Transações (cadastro completo, rollback, commit)
    \item Consultas complexas (JOINs, agregações JSON)
    \item Lookup de dados (busca de IDs, case-insensitive)
\end{itemize}

\subsubsection{Frontend-Backend}
\begin{itemize}
    \item Requisições HTTP (GET, POST), tratamento de erros
    \item Formatação de dados (JSON)
    \item Estados da aplicação (loading, error, success)
\end{itemize}

\subsection{Testes de Sistema}

\textbf{Objetivo}: Validar comportamento completo do sistema.

\textbf{Fluxos Principais}:
\begin{enumerate}
    \item Cadastro completo de beneficiário (end-to-end)
    \item Consulta e visualização de beneficiários
    \item Navegação no sistema
\end{enumerate}

\textbf{Requisitos Não-Funcionais}: Usabilidade, performance básica, segurança básica (validação de entrada, SQL Injection, CORS).

\section{Técnicas de Teste}

\subsection{Testes de Caixa Preta}

Baseados em especificações funcionais, sem conhecimento da implementação.

\textbf{Áreas de Aplicação}:
\begin{itemize}
    \item Validação de formulários (campos obrigatórios, formatos, valores válidos/inválidos)
    \item Endpoints da API (requisições válidas/inválidas, códigos HTTP, estrutura JSON)
    \item Fluxos de negócio (cadastro, busca, navegação)
    \item Regras de negócio (geração de cadastro, relacionamentos, integridade referencial)
\end{itemize}

\textbf{Exemplos de Casos}:
\begin{itemize}
    \item CT-BP-001: Cadastro com dados válidos → HTTP 201
    \item CT-BP-002: Cadastro sem campos obrigatórios → Erro de validação
    \item CT-BP-003: CPF inválido → Erro ou formatação
    \item CT-BP-004: Busca por nome → Lista filtrada
\end{itemize}

\subsection{Testes de Caixa Branca}

Baseados na estrutura interna do código, visando cobertura de caminhos.

\textbf{Áreas de Aplicação}:
\begin{itemize}
    \item Cobertura de código (funções, condições, loops, exceções)
    \item Caminhos de execução (fluxo principal, alternativo, rollback)
    \item Componentes React (renderização condicional, hooks, event handlers)
\end{itemize}

\textbf{Exemplos de Casos}:
\begin{itemize}
    \item CT-CB-001: Função getLookupId (valor encontrado, não encontrado, nulo, exceção GRAU\_PARENTESCO)
    \item CT-CB-002: Transação (sucesso completo, erro em inserção de pessoa, beneficiário, responsável → rollback)
    \item CT-CB-003: Componente React (renderização inicial, loading, sucesso, erro)
\end{itemize}

\section{Métodos de Teste}

\subsection{Classes de Equivalência}

Agrupamento de entradas em classes tratadas de forma similar.

\textbf{Exemplo: Campo CPF}

\begin{table}[h]
\centering
\caption{Classes de Equivalência - CPF}
\small
\begin{tabular}{|p{3cm}|p{5cm}|p{4cm}|}
\hline
\textbf{Classe} & \textbf{Descrição} & \textbf{Exemplos} \\
\hline
Válida & CPF com 11 dígitos & "12345678901", "111.222.333-44" \\
\hline
Inválida - Curto & Menos de 11 dígitos & "123", "123456" \\
\hline
Inválida - Longo & Mais de 11 dígitos & "123456789012345" \\
\hline
Inválida - Caracteres & Letras ou especiais & "abc12345678" \\
\hline
Inválida - Vazio & Campo vazio & "" \\
\hline
\end{tabular}
\end{table}

\textbf{Outros Campos}: Email (válido, sem @, sem domínio, vazio), Data (válida, futuro, formato incorreto, vazio), CEP (8 dígitos, curto, longo, caracteres).

\subsection{Valores de Fronteira}

Testes nos limites dos domínios de entrada.

\textbf{Exemplo: Campo Nome}

\begin{table}[H]
\centering
\caption{Valores de Fronteira - Nome}
\small
\begin{tabular}{|p{3cm}|p{5cm}|p{4cm}|}
\hline
\textbf{Teste} & \textbf{Valor} & \textbf{Resultado Esperado} \\
\hline
VF-Nome-001 & "" (vazio) & Erro: obrigatório \\
\hline
VF-Nome-002 & "A" (1 caractere) & Válido \\
\hline
VF-Nome-003 & 100 caracteres & Válido (limite) \\
\hline
VF-Nome-004 & 101 caracteres & Erro: excede limite \\
\hline
\end{tabular}
\end{table}

\textbf{Outros Exemplos}: Número de cadastro (1, 0, MAX\_INT), Lista de responsáveis (0, 1, múltiplos), Telefone (vazio, 10 dígitos, 11 dígitos, 50 caracteres, 51 caracteres).

\section{Casos de Teste Detalhados}

\subsection{Cadastro de Beneficiário}

\textbf{CT-001: Cadastro Completo com Sucesso}
\begin{itemize}
    \item \textbf{Objetivo}: Validar cadastro completo com todos os dados
    \item \textbf{Passos}: Acessar cadastro, preencher todos os campos (pessoais, documentos, saúde), adicionar responsável e membro da família, salvar
    \item \textbf{Resultado Esperado}: Mensagem de sucesso, formulário resetado, dados salvos no BD, número de cadastro gerado, relacionamentos corretos
\end{itemize}

\textbf{CT-002: Cadastro sem Campos Obrigatórios}
\begin{itemize}
    \item \textbf{Objetivo}: Validar impedimento de cadastro sem campos obrigatórios
    \item \textbf{Passos}: Deixar nome vazio, tentar salvar
    \item \textbf{Resultado Esperado}: Erro de validação, cadastro não realizado
\end{itemize}

\textbf{CT-003: Cadastro com CPF Inválido}
\begin{itemize}
    \item \textbf{Objetivo}: Validar tratamento de CPF inválido
    \item \textbf{Passos}: Preencher CPF com "123", tentar salvar
    \item \textbf{Resultado Esperado}: Erro de validação ou formatação
\end{itemize}

\subsection{Listagem e Busca}

\textbf{CT-004: Listagem de Beneficiários}
\begin{itemize}
    \item \textbf{Objetivo}: Validar exibição da lista completa
    \item \textbf{Passos}: Acessar listagem, aguardar carregamento
    \item \textbf{Resultado Esperado}: Lista exibida, ordenada por nome, colunas corretas, botões de ação visíveis
\end{itemize}

\textbf{CT-005: Busca por Nome}
\begin{itemize}
    \item \textbf{Objetivo}: Validar busca por nome
    \item \textbf{Passos}: Digitar "João" no campo de busca
    \item \textbf{Resultado Esperado}: Lista filtrada, case-insensitive, atualização em tempo real
\end{itemize}

\textbf{CT-006: Expansão de Detalhes}
\begin{itemize}
    \item \textbf{Objetivo}: Validar exibição de detalhes completos
    \item \textbf{Passos}: Clicar no botão de expandir
    \item \textbf{Resultado Esperado}: Detalhes exibidos (pessoais, documentos, saúde, responsáveis, família), botão muda para recolher
\end{itemize}

\subsection{Integração}

\textbf{CT-007: Integridade Referencial}
\begin{itemize}
    \item \textbf{Objetivo}: Validar chaves estrangeiras
    \item \textbf{Passos}: Cadastrar com cidade, raça, religião, hospital válidos, verificar no BD
    \item \textbf{Resultado Esperado}: Códigos de referência corretos, JOINs funcionando, dados relacionados exibidos
\end{itemize}

\textbf{CT-008: Transação com Rollback}
\begin{itemize}
    \item \textbf{Objetivo}: Validar rollback em caso de erro
    \item \textbf{Passos}: Simular erro durante inserção, verificar estado do BD
    \item \textbf{Resultado Esperado}: Nenhum dado parcialmente inserido, transação revertida, mensagem de erro
\end{itemize}

\section{Estratégia de Execução e Ferramentas}

\subsection{Ordem de Execução}
\begin{enumerate}
    \item Testes de Unidade (funções backend, componentes React)
    \item Testes de Integração (Backend-BD, Frontend-Backend)
    \item Testes de Sistema (fluxos completos)
\end{enumerate}

\subsection{Ferramentas}

\textbf{Jest} - Framework principal para testes de unidade e integração:
\begin{itemize}
    \item \textbf{Backend}: Jest + Supertest (testes de API)
    \item \textbf{Frontend}: Jest + React Testing Library (testes de componentes)
    \item \textbf{Cobertura}: Istanbul/nyc integrado ao Jest
\end{itemize}

\textbf{Cypress} - Framework para testes end-to-end (opcional):
\begin{itemize}
    \item Testes de sistema completos
    \item Testes de interface do usuário
    \item Validação de fluxos funcionais
\end{itemize}

\end{document}



\end{document}

